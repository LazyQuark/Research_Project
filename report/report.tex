\documentclass[a4paper,11pt]{article}

%\usepackage[hidelinks]{hyperref}
\usepackage[colorlinks=true, linkcolor=black, urlcolor=red]{hyperref}

\usepackage{amsmath}
\usepackage{amssymb}
\usepackage[margin = 1in]{geometry}
\usepackage{graphicx}
\usepackage{hyperref}

\title{\textbf{Research Project I} \\[0.5cm]
\large INDIAN INSTITUTE OF SCIENCE EDUCATION AND RESEARCH KOLKATA}


\author{KONDAPALLI VARA PRASAD \\ 22MS161 \\[0.5cm]
\textbf{Supervisor:} Prof. Rajesh Kumble Nayak}
\date{}


\begin{document}
\maketitle
\tableofcontents
\pagebreak

%%%%%%%%%%%%%%%%%%%%%%%%%%%%%

\section{Paper:I}
\begin{center}
      INTRODUCTION TO ANALYSIS OF LOW-FREQUENCY GRAVITATIONAL WAVE DATA by 
 B. F. Schutz
\end{center}

\subsection{Content :}

In this paper all data streams are treated as random, and the detection of a signal is formulated as a decision problem based on probabilities.
Each discrete data sample $x_j$ is written as a sum of a possible signal $h_j$ and noise $n_j$,
\[
x_j = h_j + n_j ,
\]
where the noise is modelled as a zero mean Gaussian process.
Its statistics are encoded in the autocorrelation
\[
K_{ij} = \mathbb{E}[n_i n_j] ,
\]
whose  Fourier transform defines the one sided noise power spectral density $S(f)$ via
\[
\langle \tilde{n}(f)\,\tilde{n}^*(f')\rangle = \tfrac{1}{2} S(f)\,\delta(f-f') .
\]
The probability densities $P_0(x)$ (noise only) and $P_1(x)$ (signal plus noise) then enter the likelihood ratio, which provides the optimal detection statistic.

\subsection{Detection statistic and matched filter}

For Gaussian noise, the log likelihood ratio for a given data stream $\{x_j\}$ can be written as
\[
\ln \Lambda(x) = \sum_k x_k q_k - \frac{1}{2} \sum_{k} h_k q_k ,
\]
where the filter (or weighting) $q_k$ is chosen so that $\ln \Lambda(x)$ is maximised for a given expected signal $h_j$.
Using the relation
\[
h_j = \sum_k K_{jk} q_k ,
\]
one obtains $q = K^{-1} h$, which in the frequency domain becomes the standard matched filter expression
\[
\tilde{q}(f) = \frac{\tilde{h}(f)}{S(f)} ,
\]
showing that each Fourier component of the template waveform is weighted by the inverse of the noise power at that frequency.

It is convenient to define the scalar detection statistic
\[
G = \sum_j x_j q_j ,
\]
which, using Parseval’s theorem, can also be expressed in terms of discrete Fourier components as
\[
G = \frac{1}{N} \sum_k \tilde{x}_k \tilde{q}_k^{*} ,
\]
and, this \[ \tilde{q}_k^{*} = N \tilde{h}_k^{*}/S_k \]
\[
G \sim \sum_k \frac{\tilde{x}_k \tilde{h}_k^*}{S_k} .
\]

\subsection{Statistical properties and SNR}

If no signal is present, so that $x_j = n_j$, the statistic $G$ is a linear combination of Gaussian noise samples and is therefore itself Gaussian with mean
\[
\mathbb{E}[G] = 0
\]
and variance
\[
\mathbb{E}[G^2] = \sum_{k=0}^{N-1} \frac{|h_k|^2}{S_k} \equiv d_0^2 .
\]
The corresponding probability density is
\[
P_0(G) = \frac{1}{\sqrt{2\pi d_0^2}} \exp\!\left(-\frac{G^2}{2 d_0^2}\right) .
\]

If a signal with spectrum $h_k$ is present, the mean of $G$ becomes
\[
\mathbb{E}[G] = d_0^2 , \qquad \mathrm{Var}(G) = d_0^2 ,
\]
and its distribution shifts accordingly:
\[
P_1(G) = \frac{1}{\sqrt{2\pi d_0^2}} \exp\!\left[-\frac{(G - d_0^2)^2}{2 d_0^2}\right] .
\]
A natural signal to noise ratio is then
\[
\mathrm{SNR} = \frac{\mathbb{E}[G]}{\sqrt{\mathrm{Var}(G)}} = \frac{d_0^2}{d_0} = d_0 ,
\]
with
\[
d_0^2 = \sum_{k=0}^{N-1} \frac{|h_k|^2}{S_k} ,
\]
which matches the familiar frequency domain SNR expression used in matched filtering.
In terms of $G$, the log likelihood ratio becomes
\[
\ln \Lambda(G) = \ln \frac{P_1(G)}{P_0(G)} = G - \frac{1}{2} d_0^2 ,
\]
so thresholding on $\ln \Lambda$ is equivalent to thresholding directly on $G$.


\begin{figure}[h]
      \centering
      \includegraphics[scale = 0.6]{Figure_1.png}
\end{figure}






\subsection{False alarms and detections}

Choosing a threshold $G_{\mathrm{thr}}$ defines a decision region
\[
R = \{x : G(x) > G_{\mathrm{thr}}\}
\]
in data space: any data set with $G > G_{\mathrm{thr}}$ is declared a candidate detection.
The false alarm probability (probability to cross threshold with noise only) is
\[
P_F = \int_{G_{\mathrm{thr}}}^{\infty} P_0(G)\, dG
      = \frac{1}{2}\,\mathrm{erfc}\!\left(\frac{G_{\mathrm{thr}}}{\sqrt{2}\, d_0}\right) ,
\]
and the detection probability (probability to cross threshold when a signal is present) is
\[
P_D = \int_{G_{\mathrm{thr}}}^{\infty} P_1(G)\, dG
      = \frac{1}{2}\,\mathrm{erfc}\!\left(\frac{d_0^2 - G_{\mathrm{thr}}}{\sqrt{2}\, d_0}\right) ,
\]
where the complementary error function is
\[
\mathrm{erfc}(z) = \frac{2}{\sqrt{\pi}} \int_z^{\infty} e^{-t^2} dt .
\]
Raising the threshold decreases $P_F$ but also tends to decrease $P_D$, so the threshold is chosen to satisfy a required false-alarm rate while retaining good detection efficiency.

\subsection{Fourier methods and sampling}

The paper also reviews how discrete Fourier transforms (DFTs) connect time domain samples $x_j$ to frequency domain values $x_k$ for a data segment of duration $T = N \Delta t$.
The discrete frequencies are
\[
f_k = \frac{k}{T}, \qquad k = 0,1,\dots,\frac{N}{2},
\]
so the frequency resolution is
\[
\Delta f = \frac{1}{T} ,
\]
and the Nyquist frequency is
\[
f_{\mathrm{Nyq}} = \frac{1}{2\Delta t} .
\]
If significant power exists above $f_{\mathrm{Nyq}}$, it will alias into lower frequencies and contaminate the measured spectrum, so the sampling rate must be high enough that all relevant signal and noise frequencies lie below $f_{\mathrm{Nyq}}$.

Parseval’s theorem relates time and frequency domain energies:
\[
\sum_{j=0}^{N-1} |x_j|^2 = \frac{1}{N} \sum_{k=0}^{N-1} |x_k|^2 .
\]
From this it follows that a coherent signal of duration $T$ grows in amplitude $\propto T$, whereas random noise grows as $\sqrt{T}$, implying
\[
\text{amplitude SNR} \propto \sqrt{T}, \qquad \text{power SNR} \propto T .
\]
The paper also discusses interpolation and zero padding in the frequency domain to obtain finer sampling of the spectrum without changing the underlying data.

\subsection{Parameter dependence}

Finally, the dependence of the detection statistic on signal parameters is handled by considering a family of templates $h_k(\boldsymbol{\theta})$ labelled by parameters $\boldsymbol{\theta} = \{\theta_1,\dots,\theta_m\}$.
For each parameter set, the optimal filter is
\[
\tilde{q}_k(\boldsymbol{\theta}) = \frac{h_k(\boldsymbol{\theta})}{S_k} ,
\]
and the corresponding statistic is
\[
G(\boldsymbol{\theta}) = \sum_k x_k\,\tilde{q}_k(\boldsymbol{\theta}) .
\]
The noise variance associated with that template is
\[
d_0^2(\boldsymbol{\theta}) = \sum_{k=0}^{N-1} \frac{|h_k(\boldsymbol{\theta})|^2}{S_k} ,
\]
and the log likelihood ratio takes the form
\[
\ln \Lambda(\boldsymbol{\theta}) = G(\boldsymbol{\theta}) - \frac{1}{2} d_0^2(\boldsymbol{\theta}) .
\]
Maximising $\ln \Lambda(\boldsymbol{\theta})$ (or equivalently $G(\boldsymbol{\theta})$ with proper normalisation) over the parameter space yields both the best matching template and the corresponding parameter estimates, providing a unified treatment of detection and parameter estimation in stationary Gaussian noise.






%%%%%%%%%%%%%%%%%%%%%%%%%%%%%%%%%%%%%%%%%%%%%%%%%%%%%%%%%%%%%%%%%%%%%%%%%%%%%%%
%################################################################################

\subsection{Mock Analysis :}

Here I have taken a signal and embedded into noise with appropriate ampltude  and the noise is generated 
from psd of the LIGO Detector and from there I assumed this is my data of time series and checked for any potential siganls 
by using multiple templates with different mass combinations. The highest SNR gives me the closest template that matched with the signal and correspondingly the closest 
parameters for the signal are obtained from the best template parameters.Throughout the Analysis my Sampling rate (sps) and Time (T) are constant for each function argument.

\begin{figure}[h]
    \centering
    \includegraphics[width=0.52\textwidth]{../noise_data.png}%
    \includegraphics[width=0.52\textwidth]{../snr_analysis.png}
\end{figure}

I have also carried out some practical coding to become familiar with the analysis of gravitational waves using the \texttt{PyCBC} package.

The corresponding Jupyter notebook is available at :
\href{https://drive.google.com/file/d/1cZhGSJQLC_IejpNyWa35ZB9D6rHQPUxg/view?usp=sharing}{Jupyter notebook}.

%%%%%%%%%%%%%%%%%%%%%%%%%%%%%%%%%%%%%%%%%%%%%%%%%%%%%%%%%%%%%%%%%%%%%%%%%%
%%%%%%%%%%%%%%%%%%%%%%%%%%%%%%%%%%%%%%%%%%%%%%%%%%%%%%%%%%%%%%%%%%%%%%%%%%
%%%%%%%%%%%%%%%%%%%%%%%%%%%%%%%%%%%%%%%%%%%%%%%%%%%%%%%%%%%%%%%%%%%%%%%%%%%
\pagebreak


\section{Paper:II}
\begin{center}
      A GRAVITATIONAL-WAVE DATA ANALYSIS PRIMER 

      by  P. Ajith -- Satya Mohapatra -- Archana Pai 
\end{center}

\subsection{Content :}

In this analysis the primary problem is to identify the gravitational waveform that is hidden in the noise.
One intutive way is to check for the signals that we are sure that it cannot be produced by noises in the time series data if we are unsure of the waveform.
It is possible to recognize much weaker signals if we know what the waveform looks like but still the parameters that define the waveform need to be known simultaneously.
Since the data has noises all over the detection of signal is based on probabilities, we need to assess this probability to have some confidence in our detection 



x$_j$ denotes a sample of the experimental data, the expected signal will be denoted by h$_j$ and the
 noise  is called n$_j$.
 In general the sample points follow a distribution function pdf: $p$($x$). 


\subsection{Newtonian chirp signal}

The starting point is a binary system of two compact stars in circular orbit, with component masses \(m_1\) and \(m_2\), which loses energy through gravitational wave emission and inspirals, producing a chirp whose amplitude and frequency increase with time. % [file:1]
In the leading Newtonian approximation, the strain is written as
\[
h(t) = A(t)\cos\bigl[\Phi(t) + \phi_0\bigr],
\]
where \(\phi_0\) is the phase when the signal enters the detector. % [file:1]
The amplitude \(A(t)\) depends on the chirp mass
\[
\mathcal{M} = \frac{(m_1 m_2)^{3/5}}{(m_1 + m_2)^{1/5}},
\]
the luminosity distance \(D\), and the detector response to the source’s sky position and orientation. % [file:1]
In an optimally oriented case, the amplitude has the characteristic Newtonian scaling
\[
A(t) \propto \frac{\mathcal{M}^{5/3}\, f(t)^{2/3}}{D},
\]
showing that more massive binaries inspiral faster and generate stronger signals at a given frequency. % [file:1]
The instantaneous gravitational wave frequency \(f(t)\) increases as the binary inspirals. % [file:1]

\subsection{Data model and statistical detection}

The detector output is modelled as
\[
x(t) = h(t) + n(t),
\]
where \(h(t)\) is the possible signal and \(n(t)\) is random noise. % [file:1]
Detection is formulated as a hypothesis test between ``noise only'' (\(x = n\)) and ``signal plus noise'' (\(x = h + n\)). % [file:1]
A likelihood ratio
\[
\Lambda(x) = \frac{P(x \mid h+n)}{P(x \mid n)}
\]
is defined, and a detection statistic (often proportional to \(\ln \Lambda\)) is compared to a threshold: if it exceeds the threshold, the data segment is declared a candidate detection. % [file:1]
The false alarm probability is the probability, under the noise only hypothesis, that this statistic crosses the threshold, and must be balanced against the detection probability when choosing the threshold. % [file:1]

\subsection{Noise and power spectral density}

The noise \(n(t)\) is assumed to be a stationary, zero mean Gaussian process, fully characterized by its two point correlation function or, equivalently, by its one sided power spectral density \(S_n(f)\). % [file:1]
In the frequency domain this is written as
\[
\big\langle \tilde n(f)\, \tilde n^{*}(f') \big\rangle = \frac{1}{2}\, S_n(f)\, \delta(f - f'),
\]
for a real time series. % [file:1]
White noise corresponds to constant \(S_n(f)\), whereas coloured noise has frequency dependent \(S_n(f)\), as in real interferometers where low and high frequencies are noisier than an intermediate band. % [file:1]

\subsection{Matched filtering and inner product}

For stationary Gaussian noise, the optimal detection method for a known waveform is matched filtering, which uses a noise weighted inner product between data and template. % [file:1]
The inner product of two real time series \(x(t)\) and \(h(t)\) is defined by
\[
(x|h) = 4 \int_{0}^{\infty} \frac{\tilde x(f)\, \tilde h^{*}(f)}{S_n(f)}\, df,
\]
so that frequencies with small \(S_n(f)\) (less noise) are weighted more strongly. % [file:1]
The matched filter output for a template \(h\) is \((x|h)\), and the optimal signal to noise ratio (SNR) is
\[
\rho = \frac{(x|h)}{\sqrt{(h|h)}},
\]
which, if the data contain exactly the template signal with some amplitude, peaks at a value proportional to that amplitude. % [file:1]

\subsection{Maximization over parameters}

The SNR is then maximized over the parameters of the Newtonian chirp. % [file:1]
The dependence on distance \(D\) appears only as an overall scale, so one uses normalized templates with \((h|h) = 1\) and recovers the amplitude (and hence \(D\)) from the measured peak SNR. % [file:1]
The unknown phase \(\phi_0\) is handled by constructing two orthogonal templates,
computing their SNRs \(\rho_c\) and \(\rho_s\), and forming a combined statistic
\[
\rho_{\max} = \sqrt{\rho_c^2 + \rho_s^2},
\]
which corresponds to analytically maximizing over \(\phi_0\). % [file:1]
The coalescence time \(t_0\) is found by the time where the peak snr value appers.
Finally, the mass dependence is covered by building a bank of templates with different chirp masses \(\mathcal{M}\), computing \(\rho_{\max}\) for each, and selecting the template that yields the highest SNR as the best match to the signal. % [file:1]









































































\end{document}